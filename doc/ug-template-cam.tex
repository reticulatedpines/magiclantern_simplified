% generated by Docutils <http://docutils.sourceforge.net/>
\documentclass[a4paper,english]{article}
\usepackage{fixltx2e} % LaTeX patches, \textsubscript
\usepackage{cmap} % fix search and cut-and-paste in PDF
\usepackage{babel}
\usepackage{mathpazo}
\usepackage{setspace}
\usepackage{MnSymbol}
\usepackage{color}
\usepackage[T1]{fontenc}
\usepackage[utf8]{inputenc}
\usepackage{ifthen}
\usepackage{float} % float configuration
\floatplacement{figure}{H} % place figures here definitely
\usepackage{graphicx}
\usepackage{tikz}
\usetikzlibrary{patterns}
\usepackage{longtable}
\usepackage{booktabs}
\usepackage{newfile}
\usepackage{ifthen}

% avoid hyphenation and a few other things 
\hyphenpenalty=1000000
\widowpenalty=1000000
\clubpenalty=1000000

\newlength{\DUtablewidth} % internal use in tables

% providelength (provide a length variable and set default, if it is new)
\providecommand*{\DUprovidelength}[2]{
  \ifthenelse{\isundefined{#1}}{\newlength{#1}\setlength{#1}{#2}}{}
}

% lineblock environment
\DUprovidelength{\DUlineblockindent}{2.5em}
\ifthenelse{\isundefined{\DUlineblock}}{
  \newenvironment{DUlineblock}[1]{%
    \list{}{\setlength{\partopsep}{\parskip}
            \addtolength{\partopsep}{\baselineskip}
            \setlength{\topsep}{0pt}
            \setlength{\itemsep}{0.15\baselineskip}
            \setlength{\parsep}{0pt}
            \setlength{\leftmargin}{#1}}
    \raggedright
  }
  {\endlist}
}{}

% wave sign for LCD remote shot
\def\wave{\tikz{ \draw [xscale=0.05,yscale=0.15] (0,0)--(1,1)--(2,0)--(3,1)--(4,0)--(5,1); }}

% unused (but maybe it will be in the future)
\def\overexposed{\tikz{\node [rectangle, pattern=north west lines, pattern color=red, minimum width=5mm, minimum height=3mm] {}; }}
\def\underexposed{\tikz{\node [rectangle, pattern=north west lines, pattern color=blue, minimum width=5mm, minimum height=3mm] {}; }}

\usepackage[colorlinks=true,linkcolor=blue,urlcolor=blue]{hyperref}
\hypersetup{
  pdftitle={Magic Lantern 0.2 for Canon 550D, Firmware 1.0.9 -- User's Guide},
  pdfauthor = {Alex Dumitrache <broscutamaker@gmail.com>},
  pdfpagelayout = {OneColumn},
  pdfpagemode = {UseNone},
  pdfstartview = {FitH},
  pdfborder = {0 0 0}
}


% page size (should result in a readable font on a 720x480 bitmap)
%\usepackage[papersize={94mm, 142mm}, total={90mm, 136mm}, centering]{geometry}
\usepackage[papersize={120mm, 80mm}, total={115mm, 73mm}, centering]{geometry}
%\usepackage[papersize={94mm, 2000mm}, total={90mm, 2000mm}, centering]{geometry}

% sans serif font (easier to read on monitors, unlike serif which is easier to read on paper)
\renewcommand{\familydefault}{\sfdefault} 

% tighter layout (camera screen is small)
\setlength{\parskip}{0mm plus1fil}
\setlength{\parindent}{0mm}

% redefine itemize and enumerate for tighter spacing, and a bit of rubber space (fil)
\let\olditemize=\itemize
\let\oldenumerate=\enumerate

\def\itemize{
\olditemize
\setlength{\itemsep}{0mm plus1fil}
\setlength{\leftskip}{-3mm}
}
\def\enumerate{
\oldenumerate
\setlength{\itemsep}{0mm plus1fil}
\setlength{\leftskip}{-3mm}
}

% create a log file for index (like a table of context in binary format)
\AtBeginDocument{\newoutputstream{menuindex}\openoutputfile{menuindex.txt}{menuindex}%
    \addtostream{menuindex}{Index for Magic Lantern context-sensitive help in menu}
}
\AtEndDocument{\addtostream{menuindex}{\arabic{page} end document}\closeoutputstream{menuindex}}%

% iq = inside quote
% if a quote environment spans on more than one page, add a "continued on next page" hint
\newcount\iq
\renewenvironment{quote}{\par\vfil\setlength{\voffset}{-1mm}\iq=1}{\vfil\setlength{\voffset}{0mm}\iq=0}

% this hint is only displayed if iq=1
\newcommand{\nextpagehint}{%
\ifthenelse{\iq=1}{%
\begin{tikzpicture}[remember picture,overlay]%
\node [yshift=-1mm] at (current page.south east) [above left, color=blue, fill=white] {... continued on next page};%
\end{tikzpicture}}{}}%

% a small ML logo
\newcommand{\cornerlogo}{%
\begin{tikzpicture}[remember picture,overlay]%
\node at (current page.north east) [below left] {\includegraphics[width=7mm]{Logo.png}};%
\end{tikzpicture}}

% ensure newpage before sections
\let\oldsection=\section
\def\section{\newpage\oldsection}

%~ \let\oldsubsection=\subsection
%~ \def\subsection{\newpage\oldsubsection}

% this is readable enough to skip the comments :)
\usepackage{everypage}
\AddEverypageHook{\nextpagehint}
%~ \AddEverypageHook{\cornerlogo}

% redefine href to show the hyperlink as plain text, since you don't have 'net on the camera :)
\let\oldhref=\href
\directlua{dofile("showlink.lua")}
\renewcommand{\href}[2]{\directlua{ShowLink("#1", "#2")}}

%~ \sloppy

% ugly...
%~ \pagecolor{black}
%~ \color{white}

%%% Body
\begin{document}
\title{\vspace{-1cm}\includegraphics[width=3cm]{Logo.png}\\Magic Lantern 0.2.1\\{\small for Canon 550D, 60D, 600D and 500D}}
\author{\url{http://magiclantern.wikia.com/unified}}
\maketitle

%~ \setlength{\baselineskip}{0mm plus1fil}

%~ \vspace{5mm}
%~ \begin{center}
%~ \setstretch{1.1}
%~ 
%~ \begin{minipage}{10cm}
%~ \input{credits.tex}
%~ \end{minipage}
%~ \vspace{5mm}
%~ 
%~ \begin{minipage}{13cm}
%~ Magic Lantern is being developed by independent film makers in our spare time and at risk to our beloved cameras. We hope that it saves you time and aggravation on set, and we'd appreciate your support. You can help by \href{https://www.paypal.com/cgi-bin/webscr?cmd=_donations&business=ELJ6U9GGFPL3U&lc=RO&item_name=Magic%20Lantern%20firmware%20for%20Canon%20550D&currency_code=EUR&bn=PP%2dDonationsBF%3abtn_donate_LG%2egif%3aNonHostedGuest}{donating via PayPal}, or through equipment donations. You can also \href{mailto:broscutamaker@gmail.com}{contact me (Alex) via email}. Thanks!
%~ 
%~ \vspace{2mm}
%~ \hskip1mm \href{https://www.paypal.com/cgi-bin/webscr?cmd=_donations&business=ELJ6U9GGFPL3U&lc=RO&item_name=Magic%20Lantern%20firmware%20for%20Canon%20550D&currency_code=EUR&bn=PP%2dDonationsBF%3abtn_donate_LG%2egif%3aNonHostedGuest}{\includegraphics[width=1.5cm]{donate.png}}
%~ \end{minipage}
%~ \end{center}

Magic Lantern is an open framework for developing enhancements to the amazing Canon digital SLRs, including, but not limited to,
5D Mark II, 60D, 500D/T1i, 550D/T2i and 600D/T3i.

Magic Lantern is free software, licensed under the GPL, brought to you by a small 
development team:
\begin{itemize}
\item Trammell Hudson -- original author and lead of Magic Lantern project
\item Arm.Indy -- author of crypto tools and virtually all new ports
\item AJ -- developer of AJ version of Magic Lantern for 5D Mark II
\item Alex -- main developer for 550D and 60D branches
\item Coutts -- developer of 500D branch
\item SztupY -- developer of 60D branch
\end{itemize}

\newpage
\subsubsection*{Thanks}

\begin{itemize}
\item Code contributions: piersg, nandoide, stefano, trho, deti, tapani, phil, xaos, Jason, AlinS, Chuchin, RoaldFre, Vincent Olivier, Colin Peart, cpc, msi, robotsound, Maclema, Adijiwa...
\item Card tools authors: Pel, Zeno, lichtjaar
\item Cropmark authors: CameraRick, Robert, bwwd, turbinicarpus, similaar, dremelv21, Piepas...
\item Tutorial authors: sawomedia, Chung Dha, Dave Dugdale, Malcolm Debono, MediaUnlocked, Jeremy, Lauren Stevens, Shayne35mm, Renny Hayes, Daniel Thurau, Dod3032, 3615geek, CineDigital.tv, jeveuxdoncjevilme...
\item All users who tested this software, provided feedback and reported bugs
\item All users who supported Magic Lantern development with donations
\item CHDK and 400plus
\end{itemize}

\newpage
%~ \tableofcontents
%~ \newpage

\def\tableofcontents{}

%~ \makeatletter
%~ \let\oldsection=\section
%~ \renewcommand{\section}[2]{\oldsection*{#2}\message{}\addtostream{menuindex}{\chaptername\arabic{page}}}

% write a computer-readable table of contents, useful for context-sensitive help
\let\oldaddcontentsline=\addcontentsline
\renewcommand{\addcontentsline}[3]{\oldaddcontentsline{#1}{#2}{#3}\addtostream{menuindex}{\arabic{page} #2 #3}}

% mkdoc-cam.py will write here the result from rst2latex
$body


\end{document}


